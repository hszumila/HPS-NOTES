\documentclass[12pt]{article}
        \usepackage{graphicx,type1cm,eso-pic,color}
        \usepackage{hyperref}
        \usepackage[left=3.0cm,right=3.0cm,top=2cm,bottom=2cm,headheight=13.6pt]{geometry}
        \usepackage{subfigure}

\makeatother


\title{Manual for HPS ECal v1.3}

\author{ECal cell phone: 757-810-1489 (ECal cell phone) \\ 
Authors: \\
General contact: Rapha\"el \textsc{Dupr\'e} (dupre@ipno.in2p3.fr)\\ 
LED system: Andrea \textsc{Celentano} (andrea.celentano@ge.infn.it)\\
LV/HV supply and chiller: Nathan \textsc{Baltzell} (baltzell@jlab.org)\\
}

\date{\today} %01/07/2014

\begin{document}
\maketitle{}

   \section{General description of the ECal}


The electromagnetic calorimeter (ECal), installed downstream of the pair spectrometer dipole magnet (figure~\ref{GView}), performs two essential functions for the experiment: it provides the trigger signal and helps identify electrons and positrons. The ECal modules are based on tapered 160 mm long PbWO crystal with a 13.3x13.3 mm$^2$ (16x16 mm$^2$) front (rear) face wrapped in VM2000 multilayer polymer mirror film. The scintillation light, approximately 110 photons / MeV, is read out by a 10x10 mm$^2$ Hamamatsu S8664-1010 Avalanche Photodiode (APD) with 75\% quantum efficiency glued to the rear face surface. The low gain of APDs (150 pC/pC) is compensated with custom-made preamplifier boards, which provide a factor of 225 amplification of the APD signal. In front of the crystals, LEDs are installed to send light into the crystals. These are used in order to check the proper functioning of the ECal and provides complementary information to evaluate gain variations in the various channels of the calorimeter (see figure~\ref{AmplChain}).

\begin{figure}[hp]
\center
\includegraphics[width=0.75\textwidth]{GView.png}
\caption{\small \label{GView} General view of the ECal (in color) suspended at the downstream end of the HPS analyzing magnet.}
\end{figure}

\begin{figure}[hp]
\center
\includegraphics[width=0.75\textwidth]{CrystalAssembly.png}
\caption{\small \label{AmplChain} View of an ECal crystal and the amplification chain.}
\end{figure}
      

\begin{figure}[hp]
\center
\includegraphics[width=0.75\textwidth]{ECal2.png}
\caption{\small \label{Crystals} Front view of the ECal crystals layout.}
\end{figure}
      
The ECal is built in two separate halves that are mirror reflections of one another relatively to the horizontal plane. The 221 modules in each half are supported by aluminum frames and arranged in rectangular formation with five layers and 46 crystals / layer, except for the layer closest to the beam where nine modules were removed to allow a larger opening for the outgoing electron and photon beams (figure~\ref{Crystals}). Each half is enclosed in a temperature controlled box ( $< 1^\circ$F stability and $< 4^\circ$F uniformity) to stabilize the crystal light yield and the operation of the APDs. Four printed circuit boards (referred as mother boards) mounted on the back plane penetrate the enclosure and are used to supply the $\pm5$ V operating voltage for the preamplifiers, the 400 V bias voltage to the APDs, and to read out signals from the APDs. Each half of the ECal is divided into 26 bias voltage groups formed in order to minimize the gain spread of the APD-preamplifier couples.


After a 2:1 signal splitter, 1/3 of an amplified APD signal is fed to a single channel of a JLab flash ADC (FADC) board. 2/3 of the signal is sent to a discriminator module before a TDC for a time measurement. The FADC boards are high speed VXS modules digitizing up to 16 crystal signals at 250 MHz and storing 4 ns samples with 12-bit resolution. When a trigger is received, the pipeline is read on these boards from 5 samples before and 30 after the trigger time (those values will be adapted during commissioning).

\newpage
 \twocolumn
\part{Shift Takers Instructions}

\vspace*{\stretch{1}}      
Most ECal controls are accessible through EPICS, from the main window (figure~\ref{EPICSmain}). From there you can access {\bf Temperature monitoring} in {\it Miscellaneous} then {\it ECal Temperature}, the {\bf ECal chiller} in {\it Devices} then {\it Chiller (ECAL)}, the {\bf Scalers} in {\it ECal Scaler GUI}, the {\bf ECal high voltage} in {\it Voltages} then {\it ECal HV} and the {\bf LED control panel} in {\it Devices} then {\it Flasher}.
\vspace*{\stretch{1}}      

\begin{figure}[h!]
\center
\includegraphics[width=0.38\textwidth]{hps_epics_2014_12_15.png}
\caption{\small \label{EPICSmain} View of the Hall-B EPICS main window.}
\end{figure}

 \onecolumn

      \section{Temperature}

         The ECal temperature should remain as stable as possible in order to avoid gain variation in the system. Eighteen temperature sensors are placed in the ECal enclosure and should be monitored through EPICS (see figure~\ref{temp} and~\ref{temp2}). Variations of two degrees F or more during a shift should be reported to ECal expert on call and noted in the log book.

\begin{figure}[htbp]
\center
\includegraphics[width=0.65\textwidth]{EcalTemp_2014_12_20.png}
\caption{\small \label{temp} View of the EPICS temperature monitoring window.}
\end{figure}
      
\begin{figure}[htbp]
\center
\includegraphics[width=0.95\textwidth]{ECal_temp_s.png}
\caption{\small \label{temp2} View of the EPICS temperature monitoring strip charts.}
\end{figure}

       \section{Chiller}

         The chiller allows to keep the calorimeter at the right temperature and should be ON and set at 17C at all times. The chiller can be monitored through its webcam (figure~\ref{ChillerCam}) or using its EPICS controls (figure~\ref{ChillerEPICS}). Shift takers should not attempt to change the chiller settings and call ECal expert in case of problem.

\begin{figure}[h]
\center
\includegraphics[width=0.75\textwidth]{ChillerCam_2014_12_20.png}
\caption{\small \label{ChillerCam} View of the chiller screen by webcam (cctv10.jlab.org).}
\end{figure}

\begin{figure}[h]
\center
\includegraphics[width=0.39\textwidth]{ChillerWin_2014_12_20.png}
\caption{\small \label{ChillerEPICS} View of the EPICS ECal Chiller window.}
\end{figure}


      \section{Scalers}

         Rates seen by the ECal are available in the EPICS (Fig \ref{Scalers}), they represent the rates as seen from the FADC and TDC electronics. The difference is mainly due to their different thresholds. One can also see scalers from the DAQ GUI (figure~\ref{DAQscalers}), this indicates the rates of clusters reconstructed by the trigger electronics. These numbers should all remain constant within ~10\% during stable beam operation. A strong increase is the indication of bad beam conditions or is due to the presence of a new source of noise, in the latter case, please contact ECal expert on call.

\begin{figure}[hbp]
\center
\includegraphics[width=0.75\textwidth]{ECAL_FADC_SCALER_2014_12_20.png}
\caption{\small \label{Scalers} View of the EPICS FADC and DSC2 scalers window.}
\end{figure}

\begin{figure}[hbp]
\center
\includegraphics[width=0.75\textwidth]{ecal-cluster-12-20-14.png}
\caption{\small \label{DAQscalers} View of the DAQ scaler window.}
\end{figure}

   \section{High Voltages}

      \subsection{Low Voltage Controls}

      The low voltage power supply must be on before HV. It is controlled manually in the hall and should be monitored using its webcam (figures~\ref{LVCam}). Call the ECal expert if this appears not to be ON or shows an abnormal current.

\begin{figure}[htbp]
\center
\includegraphics[width=0.75\textwidth]{LVCam_2014_12_20.png}
\caption{\small \label{LVCam} View of the LV screen by webcam (cctv11.jlab.org).}
\end{figure}

      \subsection{Turning ON High Voltages}

      The high voltage supply of the ECal is controlled and monitored using the EPICS application (see figure~\ref{HV}). 

\begin{figure}[htbp]
\center
\includegraphics[width=0.95\textwidth]{ecalhv_2014_12_15_16:02:54.png}
\caption{\small \label{HV} View of the EPICS ECal HV monitoring window.}
\end{figure}

   \subsection{Responding to HV trips}

      HV problems, in particular trips, are indicated by a red group in the main EPICS GUI (figure~\ref{HV}). Record all HV trips in the log book with indication of the group and run number concerned. HV can be turned back on in the EPICS HV control screen (figure~\ref{HVControl}) accessed in the main EPICS GUI. N.B. The HV can take up to 3 minutes to turn back on so you should end the current run and begin a new one when the high voltage is back on. If you cannot get a HV group to work contact the ECal expert on call.

      {\bf If you encounter more than two HV trips during your shift for the same group, you should notify the ECal Expert.}

\begin{figure}[htbp]
\center
\includegraphics[width=0.85\textwidth]{ecalhv_setting_2014_12_15.png}
\caption{\small \label{HVControl} View of the EPICS ECal HV control window.}
\end{figure}

   \section{LED Monitoring}

      \subsection{System operations - EPICS GUI}
      
      The LED system is operated trough an EPICS GUI, that is accessible trough the main HPS EPICS menu, trough Devices, then Flasher (see Figure \ref{FlasherMEDM}).

Shift takers are requested to operate the system in ``Sequence mode'' only. To do so, when requested, click on ``Initialize Flasher'', then verify the TOP frequency is 8000 Hz, and if necessary adjust it trough the proper drop-down menu. Finally, to start the sequence, click on ``Start Blue Seq'' (to use blue LEDs) or ``Start Red Seq'' (to use red LEDs). During such a run the DSC scaler screen allows to check the proper functioning of the channels (figure~\ref{LEDScalers}). 

\begin{figure}[htbp]
\center
\includegraphics[width=0.75\textwidth]{FlasherMEDM.png}
\caption{\small \label{FlasherMEDM} The HPS-ECAL Led monitoring system EPICS GUI.}
\end{figure}
\begin{figure}[htbp]
\center
\includegraphics[width=0.95\textwidth]{DSCScalersLED_2014_12_20.png}
\caption{\small \label{LEDScalers} The HPS DSC scaler during a LED run.}
\end{figure}

      
   \section{Making a cosmic calibration run}

   To be added later.

\newpage

\part{ECal Experts Resources}

   \section{Localization of ECal elements for experts}

{\bf REMINDER:} Since the ECal is within 3 feet of the beam line it needs to be surveyed by RADCON before any work can be done on it.

{\it Location of elements (electronics, chiller...) in the Hall to be added with images.}


   \section{Cooling system}

     The cooling system is using a xxxx chiller that is controlled through EPICS (Fig to be added). The setting should not be modified, the temperature setting should be fixed at 17 degrees Celsius. In case of problem with the chiller contact ??? (who can take care of these in Hall-B engineer group?).

     {\it Add basic information to reset the Chiller. Add link to manual.}

   \section{Changing LV settings}
      Low voltage power supply should be set at $\pm5$V. The low voltage supply might have difficulties to get at this level because of the high current. If that was the case check, with all power supplies off, that all connection are goods. Then contact run coordinator to see if LV power supply addition is possible. 

   \section{Changing HV settings}
      {\bf NOTE:} Changing voltage settings should be taken care of in coordination with the ECal group (contact R.~Dupre). Current setting can be increased in case of need, please document this change in the log book and notify the ECal expert on call.

 {\bf NOTE:} The ECal HV groups had to be renumbered in the EPICS, the correspondence map (figure~\ref{ExpertMap}) is available in the main ECal HV monitoring window with the Expert HV Map button.

\begin{figure}[htbp]
\center
\includegraphics[width=0.95\textwidth]{ecalhv_expertmap_2014_12_15.png}
\caption{\small \label{ExpertMap} Expert HV channel map for reference.}
\end{figure}

      If for some reason some channels were to drop in gain (or increase) or if the current drawn increases in a group, it might be necessary to change the HV settings in the expert ECal EPICS control (Fig.~\ref{EHV}). A modification of the voltage will lead to a modification of the gain used by the trigger system, these values need to be updated at the same time!

\begin{figure}[htbp]
\center
\includegraphics[width=0.85\textwidth]{ecalhv_parameters_2014_12_15.png}
\caption{\small \label{EHV} View of the EPICS HV expert control window. It is accessed from the parameters button in the ECal HV control screen \ref{HVControl}}
\end{figure}

   \section{Long term HV monitoring}

{\it Add here commands to make current plot of fig~\ref{HVhistory}}

\begin{figure}[htbp]
\center
\includegraphics[width=0.95\textwidth]{ECALHVCURRENTS_2014_12_20.png}
\caption{\small \label{HVhistory} Expert HV current history.}
\end{figure}

   \section{Disconnection of a Channel and Preamplifier Replacement}
     
      In last resort, to recover a HV group that is tripping one can disconnect the faulty channel causing trouble. To do so, you need to find exactly which channel is involved! It might be obvious from data, if the channel was already very noisy, else you will have to test the channels of the group one by one. This is a lengthy operation and should only be attempted with the authorization of the run coordinator and in coordination with the ECal Group. It necessitates that the Hall-B crew moves the ECal out of the beam line and to open it.

   \section{LED system for experts}

This section has to be replaced with instructions to use the CLAS\_css GUI.

\begin{figure}[htbp]
\center
\includegraphics[width=0.85\textwidth]{LEDExpert_2014_12_20.png}
\caption{\small \label{LEDexpert} View of the LED expert controls.}
\end{figure}

\end{document}
